\documentclass[spec, och, labwork]{SCWorks}
\usepackage{preamble}

\title{Отчёт по практическим заданиям по курсу <<Нейронные сети>>}
\author{Гущина Андрея Юрьевича} % Фамилия, имя, отчество в родительном падеже

\begin{document}
\selectlanguage{russian}
\chair{теоретических основ компьютерной безопасности и криптографии}
\course{4}
\group{431}
\department{факультета компьютерных наук и информационных технологий}
\napravlenie{10.05.01 "--- Компьютерная безопасность}

% Предмет для labwork2
% \subject{} 

% Для студентки. Для работы студента следующая команда не нужна.
% \studenttitle{Студентки}

% Заведующий кафедрой
\chtitle{д.ф.-м.н., доцент} % степень, звание
\chname{М.~Б.~Абросимов}

%Научный руководитель (для реферата преподаватель проверяющий работу)
\satitle{доцент} %должность, степень, звание
\saname{И.~И.~Слеповичев}

% Руководитель практики от организации (только для практики,
% для остальных типов работ не используется)
% \patitle{к.ф.-м.н.}
% \paname{С.~В.~Миронов}

% Семестр (только для практики, для остальных
% типов работ не используется)
%\term{8}

% Наименование практики (только для практики, для остальных
% типов работ не используется)
%\practtype{преддипломная}

% Продолжительность практики (количество недель) (только для практики,
% для остальных типов работ не используется)
%\duration{4}

% Даты начала и окончания практики (только для практики, для остальных
% типов работ не используется)
%\practStart{30.04.2019}
%\practFinish{27.05.2019}

% Год выполнения отчета
\date{\the\year{}}

\maketitle

% Включение нумерации рисунков, формул и таблиц по разделам
% (по умолчанию - нумерация сквозная)
% (допускается оба вида нумерации)
% \secNumbering

\tableofcontents

\intro

Задания выполнены на языке программирования Rust c использованием библиотек
\begin{itemize}
  \item \verb|clap|
  \item \verb|serde|
  \item \verb|serde_json|
  \item \verb|burn|
  \item \verb|xml|
\end{itemize}

Для сборки задания $N$ необходимо перейти в директорию \texttt{nntaskN} и
выполнить команду
\begin{minted}[breaklines,fontsize=\small]{text}
cargo build --release
\end{minted}

В результате сборки загрузятся соответствующие библиотеки и появится директория
\verb|target| в текущей директории. В ней будет находиться исполняемый файл
\texttt{nntaskN} (или \texttt{nntaskN.exe} на Windows). Примеры запуска каждого
из заданий приведены ниже.

\section{Создание ориентированного графа}

\subsection*{Вход}

Текстовый файл с описанием графа в виде списка дуг:
\begin{equation*}
    (a_1, b_1, n_1), (a_2, b_2, n_2), \dots, (a_k, b_k, n_k),
\end{equation*}
где $a_i$ --- начальная вершина дуги $i$, $b_i$ --- конечная вершина дуги $i$,
$n_i$ --- порядковый номер дуги в списке всех заходящих в вершину $b_i$ дуг.

\subsection*{Выход}

\begin{itemize}
  \item
    Ориентированный граф с именованными вершинами и линейно упорядоченными
    дугами (в соответствии с порядком из текстового файла), либо
  \item Сообщение об ошибке в формате файла, если ошибка присутствует.
\end{itemize}

\subsection*{Пример исполнения программы}

Пусть задан файл \verb|tests/t1_input.txt| со следующим содержимым:
\inputminted{text}{../tests/t1_input.txt}

При запуске программы командой
\begin{minted}[breaklines,fontsize=\small]{text}
./nntask1 --input1 ./tests/t1_input.txt --output1 ./tests/t1_output.xml
\end{minted}
в результате получим файл \verb|tests/t1_output.xml|
\inputminted{xml}{../tests/t1_output.xml}


\section{Создание функции по графу}

\subsection*{Вход}

Ориентированный граф с именованными вершинами как описано в задании 1.

\subsection*{Выход}

\begin{itemize}
  \item
    Линейное представление функции, реализуемой графом в префиксной скобочной
    записи, либо
  \item Сообщение об ошибке в формате файла, если ошибка присутствует, либо
  \item Сообщение о наличии в графе циклов.
\end{itemize}

\subsection*{Пример исполнения программы}

Пусть задан файл \verb|tests/t1_output.xml| со следующим содержимым:
\inputminted{xml}{../tests/t1_output.xml}

При запуске программы командой
\begin{minted}[breaklines,fontsize=\small]{text}
./nntask2 --input1 ./tests/t1_output.xml --output1 ./tests/t2_output.txt
\end{minted}
в результате получим файл \verb|tests/t2_output.txt|
\inputminted{xml}{../tests/t2_output.txt}


Если в качестве ввода задан граф с циклом (файл \verb|tests/t2_input_cycle.xml|):
\inputminted{xml}{../tests/t2_input_cycle.xml}

То при запуске программа выведет сообщение:
\begin{minted}[breaklines,fontsize=\small]{text}
Некорректный ввод - в графе есть циклы
\end{minted}


\section{Вычисление значения функции на графе}



\subsection*{Вход}

\begin{enumerate}
  \item
    Текстовый файл с описанием графа в виде списка дуг (задание 1).
  \item
    Текстовый файл соответствий арифметических операций именам вершин:
    \begin{minted}[]{json}
{
    "a_1": "операция_1",
    "a_2": "операция_2",
    ...,
    "a_n": "операция_n",
}
    \end{minted}
    где a_i --- имя i-й вершины, операция_i -- символ операции, соответствующий
    вершине a_i.

    Допустимы следующие символы операций:
    \begin{itemize}
      \item <<+>> --- сумма значений,
      \item <<*>> --- произведение значений,
      \item <<exp>> --- экспонирование входного значения,
      \item <<число>> --- любая числовая константа.
    \end{itemize}
\end{enumerate}

\subsection*{Выход}

Значение функции, построенной по графу (1) и файлу (2).

\subsection*{Пример исполнения программы}

Пусть задан файл \verb|tests/t1_output.xml| с некоторым графом:
\inputminted{xml}{../tests/t1_output.xml}

Также задан файл \verb|tests/t3_ops.json| с соответствием операций вершинам:
\inputminted{json}{../tests/t3_ops.json}

При запуске программы командой
\begin{minted}[breaklines,fontsize=\small]{text}
./nntask3 --input1 ./tests/t1_output.xml --input2 ./tests/t3_ops.json --output1 ./tests/t3_output.txt
\end{minted}
в результате получим файл \verb|tests/t3_output.txt|
\inputminted{xml}{../tests/t3_output.txt}


\section{Построение многослойной нейронной сети}

\subsection*{Вход}

\begin{enumerate}
  \item
    Текстовый файл с набором матриц весов межнейронных связей в формате:
    \begin{minted}[]{json}
{
    "weights": [
        [
            [M1_11, M1_12, ..., M1_1n],
            ...
            [M1_m1, M1_m2, ..., M1_mn]
        ],
        ...,
        [
            [Mp_11, Mp_12, ..., Mp_1n],
            ...
            [Mp_m1, Mp_m2, ..., Mp_mn]
        ]
    ]
}
    \end{minted}
  \item
    Текстовый файл с входным вектором в формате:
    \begin{minted}[]{text}
x_1, x_2, ..., x_n
    \end{minted}
\end{enumerate}

\subsection*{Выход}

\begin{enumerate}
  \item
    Файл с выходным вектором – результатом вычислений НС в формате:
    \begin{minted}[]{text}
y_1, y_2, ..., y_n
    \end{minted}
  \item
    Сообщение об ошибке, если в формате входного вектора или файла описания НС
    допущена ошибка.
\end{enumerate}

\subsection*{Пример исполнения программы}

Пусть задан файл \verb|tests/t4_w.json| с весами НС:
\inputminted{xml}{../tests/t4_w.json}

Также задан файл \verb|tests/t4_x.txt| с входным вектором:
\inputminted{json}{../tests/t4_x.txt}

При запуске программы командой
\begin{minted}[breaklines,fontsize=\small]{text}
./nntask3 --input1 ./tests/t4_w.json --input2 ./tests/t4_x.txt --output1 ./tests/t4_output.txt
\end{minted}
в результате получим файл \verb|tests/t4_output.txt|
\inputminted{xml}{../tests/t4_output.txt}


\end{document}
